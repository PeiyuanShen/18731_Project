\section{Conclusions and Limitations}

We presented a security policy synthesizer for spring web applications.
%
Our security policy synthesizer can automatically synthesis security policies
for each of the API in the web application.
%
Our synthesizer takes three steps to generate the policies: security-sensitive
events identification, calling-context topology construction, and security
policy generation.
%
We applied our security policy synthesizer on \lancie and manually checked the
policies generated and detected security vulnerabilities in this application.
%
Results show that \lancie has a few vulnerabilities in access control for
different roles.

To further extend this work, the next step of this project is to generalize our
security policy synthesizer to be applicable for all spring web applications.
%
Currently, we limited our application selection to \lancie due to time
constrain and the security-sensitive events detection is based on our
observation to \lancie.
%
Another thing we plan to do in the future is to automatically generate the
vulnerability report to allow the process to scale to more applications.
%
We also plan to add more fields for statements in each function to potentially
recognize more types of vulnerabilities.
