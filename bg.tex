\section{Background and Related Work} %1 page
% what’s proposed already? What have others done already? What did they learn?
% taxonomies prior work and highlight what is new/different in your work vs
% theirs
In this section, we give an overview of one of the most popular web application
framework, Spring~\cite{spring+security:home}, and review prior work related to
our projects.

\subsection{Spring}

Spring~\cite{spring+security:home} is said to be the most popular Java web
application framework.
%
The Spring Framework is an application framework and inversion of control
container for the Java platform.
%
Spring Security~\cite{spring+security:method} is a powerful authentication and
access-control framework, which is the standard for securing Spring-based
applications.
%
Spring security provides comprehensive and extensible support for both
authentication and authorization, and prevents attacks like session fixation,
clickjacking, etc.

\subsection{Java Program Analysis}

An access control policy is an expressive specification of what resources can be
accessed, by whom, and under what conditions.
%
Bakers \etal~\cite{Backes+etal:2018:Zelkova} presents a formalization of the
Amazon Web Services (AWS) policy language  that lets users govern access to AWS
resources. This paper also presents an analysis tool, ZELKOVA, for verifying
policy properties.

Static information-flow analysis (especially taint-analysis) is a key technique
to compute where sensitive or untrusted data can propagate in a program.
Points-to analysis~\cite{10.1145/3133926} is a fundamental static program
analysis, computing what abstract objects a program expression may point to.

Bravenboer \etal~\cite{Bravenboer:2009:Doop} presented Doop\footnote{Publicly
  available at: \url{http://doop.program-analysis.srg/}.} which is a framework
for Java pointer and taint analysis.
%
Doop specifies pointer analysis algorithms declaratively using Datalog: a
logic-based language for defining recursive relations.
%
At its core, Doop is a collection of various analyses expressed in the form of
Datalog rules.

Antoniadis \etal~\cite{Antoniadis+etal:2020:Java} proposed a static
analysis framework called JackEE for Java Application, and introduced a
sound-modulo-analysis model of Java HashMap to improve precision and
scalability. JackEE can analyze both Servlets and Spring APIs. JackEE is also
evaluated through realistic benchmarks. It is proved to have a higher
reachability than some other popular tools like Doop.

Tripp \etal~\cite{10.1145/1542476.1542486} designed a static Taint Analysis for
Java (TAJ) that can analyze applications of any size, especially industry-length
programs. TAJ has the techniques to handle reflective calls, flow through
containers, nested taint, and issues in generating useful reports.

Lerch \etal~\cite{10.1145/2635868.2635878} presented FlowTwist, a taint-analysis
approach that works inside-out. They expose a design of the analysis approach
based on the IFDS algorithm and several extensions to IFDS that enable reporting
of error situations to security analysts.

Sridharan \etal~\cite{10.1145/2076021.2048145} added framework-for-framework
support to a state-of-the-art taint-analysis engine, which is useful for web
applications that utilize one or more web frameworks.

Mues \etal~\cite{10.1007/978-3-030-63461-2_7} presented JAINT, a generic security
analysis for JAVA Web-applications in which dynamic taint analysis is
integrated. They also design a domain-specific language for users to specify
taint-based security analyses for applications.

Rather than taking a specification of this logic as input, RoleCast
\etal~\cite{10.1145/2048066.2048146} is a system that statically identifying
security logic that mediates security-sensitive events in Web applications. It
identifies security-critical variables and applies role-specific variable
consistency analysis to find missing security checks. RoleCast performs well for
PHP and JSP applications.

Arzt \etal~\cite{10.1145/2594291.2594299} presented FlowDroid, a static taint
analysis for Android applications, which makes use of Android's precise
lifecycle to handle callbacks, and context, flow, field and object-sensitivity
to minimize faults. They proposed DroidBench, for Android applications
taint-analysis evaluation.

Yang \etal~\cite{6394931} proposed an approach called
Leak Miner for detecting information leakage on Android applications on market
site before users download them, which avoids runtime overhead to normal usage.

Pauck \etal~\cite{10.1145/3236024.3236029} presented ReproDroid, a framework
allowing the accurate comparison of Android taint analysis tools, inferring the
ground truth for data leaks in apps, in automatically applying tools to
benchmarks, and in evaluating the obtained results.

Ujcich \etal~\cite{Ujcich+etal:2020:EventScope} in 2020 proposed a technique
called EventScope to extend the vulnerability identification~\cite{6994333} to
SDN. EventScope automatically analyzes SDN control plane event usage, discovers
candidate vulnerabilities based on missing event-handling routines, and
validates vulnerabilities based on data plane effects.

